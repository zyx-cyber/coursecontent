\documentclass[a4paper ,12pt]{article}
\usepackage{ctex}
\usepackage{hyperref}
\usepackage{graphicx}

\begin{document}
  \title{系统开发工具基础课程实验报告}
  \author{姓名:张誉馨}
  \date{\today}
  \maketitle
  
  \pagenumbering{roman}
  \tableofcontents
  \newpage
  \pagenumbering{arabic}
  
  \section{练习内容和结果}
  
  \subsection{Git}
  1.克隆课程网站的仓库
  \begin{figure}[h]
  	\centering
  	\includegraphics[width=0.5\textwidth]{p1}
  	\caption{克隆仓库}
  \end{figure}
  
  2.将版本历史可视化并进行探索
  
  git log - -all - -graph - -decorate
  \begin{figure}[h]
  	\centering
  	\includegraphics[width=0.5\textwidth]{p2}
  	\caption{版本历史可视化}
  \end{figure}
  
  3.是谁最后修改了 README.md 文件?
  
  git log -1 README.md
  
  (-x 选项:查看最新的 x 次提交或特定文件的版本信息)
  \begin{figure}[h]
  	\centering
  	\includegraphics[width=0.5\textwidth]{p3}
  	\caption{查看最后修改}
  \end{figure}
  
  4.最后一次修改\_config.yml 文件中 collections:行时的提交信息是什么?
  
  git blame \_config.yml $|$ grep collections
  
  git show - -pretty=format:"\%s" a88b4eac $|$ head -1
  \begin{figure}[h]
  	\centering
  	\includegraphics[width=0.5\textwidth]{p4}
  	\caption{输入命令查看提交信息}
  \end{figure}
  \begin{figure}[h]
  	\centering
  	\includegraphics[width=0.5\textwidth]{p5}
  	\caption{输出提交信息}
  \end{figure}
  
  5.使用 Git 时的一个常见错误是提交本不应该由 Git 管理的大文件,或是将含有敏感信息的文件提交给 Git 。尝试向仓库中添加一个文件并添加提交信息,然后将其从历史中删除。
  
  首先提交一些敏感信息
  
  echo "password123"$>$my\_password
  
  git add .
  
  git commit -m "add password123 to file"
  
  git log HEAD
  \begin{figure}[h]
  	\centering
  	\includegraphics[width=0.5\textwidth]{p6}
  	\caption{提交敏感信息}
  \end{figure}
  
  6.接上一问,将其从历史中删除。
  
  git filter-branch --force --index-filter\
  
  'git rm --cached --ignore-unmatch ./my\_password' \
  
  - -prune-empty - -tag-name-filter cat - - { } - -all
  \begin{figure}[h]
  	\centering
  	\includegraphics[width=0.5\textwidth]{p7}
  	\caption{从历史中删除}
  \end{figure}
  
  7.从 GitHub 上克隆某个仓库,修改一些文件。当您使用 git stash 会发生什么?
  
  8.接上一问,当您执行 git log –all –oneline 时会显示什么?
  
  9.接上一问,通过 git stash pop 命令来撤销 git stash 操作
  \begin{figure}[h]
  	\centering
  	\includegraphics[width=0.5\textwidth]{p8}
  	\caption{git stash 以及撤销}
  \end{figure}
  
  10.与其他的命令行工具一样,Git 也提供了一个名为 ~/.gitconfig 配置文件 (或 dotfile)。请在 ~/.gitconfig 中创建一个别名,使您在运行 git graph 时,您可以得到 git log --all --graph --decorate --oneline 的输出结果
  
  11.分解上一问,首先打开 ~/.gitconfig文件
  
  nano ~/.gitconfig
  \begin{figure}[h]
  	\centering
  	\includegraphics[width=0.5\textwidth]{p9}
  	\caption{打开~/.gitconfig文件}
  \end{figure}
  
  12.继续分解第10问,11问之后在打开的文件中添加别名配置,将以下内容添加到文件中
  
  [alias]
  
  graph = log - -all - -graph - -decorate - -oneline
  \begin{figure}[h]
  	\centering
  	\includegraphics[width=0.5\textwidth]{p10}
  	\caption{在打开的文件中添加别名配置}
  \end{figure}
  
  13.继续10问,运行git graph,输出结果
  \begin{figure}[h]
  	\centering
  	\includegraphics[width=0.5\textwidth]{p11}
  	\caption{运行git graph}
  \end{figure}
  
  14.继续10问,运行git log --all --graph --decorate --oneline,并与运行git graph之后的结果进行比对
   \begin{figure}[h]
  	\centering
  	\includegraphics[width=0.5\textwidth]{p12}
  	\caption{运行git log - -all - -graph - -decorate - -oneline}
  \end{figure}
  
  发现二者运行结果一致
  
  15.您可以通过执行 git config –global core.excludesfile ~/.gitignore\_global 在 ~/.gitignore\_global 中创建全局忽略规则。配置您的全局 gitignore 文件来自动忽略系统或编辑器的临时文件,例如 .DS\_Store
  
  git config - -global core.excludesfile \textasciitilde/.gitignore .DS\_Store
  \begin{figure}[h]
  	\centering
  	\includegraphics[width=0.5\textwidth]{p13}
  	\caption{创建全局忽略规则}
  \end{figure}
  
  16.设置用户名和邮箱
  
  git config --global user.name "你的用户名"
  
   git config --global user.email "你的邮箱地址"
  
  替换 "你的用户名" 和 "你的邮箱地址" 为你实际的用户名和邮箱地址。
  
  --global 选项表示这些设置将应用于所有Git仓库。如果你只想对当前仓库设置不同的用户名和邮箱,可以去掉 --global 选项。
  
  17.验证设置:可以通过以下命令检查设置是否成功:
  
  git config - -global user.name
  
  git config - -global user.email
  \begin{figure}[h]
  	\centering
  	\includegraphics[width=0.5\textwidth]{p14}
  	\caption{验证设置的用户名和邮箱}
  \end{figure}
  
  \subsection{LaTeX}
  1.构建文档框架
  \begin{figure}[h]
  	\centering
  	\includegraphics[width=0.5\textwidth]{p16}
  	\caption{构建文档框架}
  \end{figure}
  
  2.添加文档标题
  \begin{figure}[h]
  	\centering
  	\includegraphics[width=0.5\textwidth]{p15}
  	\caption{添加文档标题}
  \end{figure}
  
  3.添加文档章节
  \begin{figure}[h]
  	\centering
  	\includegraphics[width=0.5\textwidth]{p17}
  	\caption{添加文档章节}
  \end{figure}
  
  4.在文档中创建目录
  
  在 maketitle 之后输入图18中所示内容:
  \begin{figure}[h]
  	\centering
  	\includegraphics[width=0.5\textwidth]{p18}
  	\caption{在文档中创建目录}
  \end{figure}
  
  5.中文字体支持
  
  只需要在文档的前导命令部分添加:
  \begin{figure}[h]
 	\centering
 	\includegraphics[width=0.5\textwidth]{p19}
 	\caption{中文字体支持}
  \end{figure}
  
  6.特殊字符
  
  \begin{figure}[h]
  	\centering
  	\includegraphics[width=0.5\textwidth]{p20}
  	\caption{特殊字符}
  \end{figure}
  例如下图中\_需要添加转义字符
  
  
  7.添加图表
  
  这里我们需要引入 graphicx 包。图片应当是 PDF,PNG,JPEG 或者 GIF 文件,详情请见文档末尾的github链接
  \section{解题感悟}
  这次实验对我来真的是一场洗礼了,它让我接触了Git、Github以及Latex排版系统。一开始面对这个实验的时候我是一无所知的,甚至有些手足无措。然后我就开始探索,先是主动跑过去请教了助教和老师这大概是个什么东西,之后开始搜索,找到了加速访问github网站的方法,即通过Microsofe Store里面的watt toolkit来加速访问,当我可以访问github的时候,说实话心里很激动。
  
  之后我开始考虑git的相关操作,我在虚拟机里试着在终端里操作git,发现不行,因为访问不到github,试了n次,搜索了n个方法,无果。于是,我开始试着在windows里面下git,这个是在csdn里面搜的文章里面有安装教程和链接,这个还比较好下。
  
  之后我开始下载Latex,下载过程可谓曲折坎坷,无比痛苦,此处省略一万字。我下了一遍又一遍,都说texstudio无法识别已经下载的texlive,最后实在没有办法了,我开始向同学请教,A同学把他的安装包试了好几种办法终于传给了我,然而还是不识别!!我的这个电脑一向太过高冷、太过谨慎,之前下载codeblocks的时候也是历经了千辛万苦。
  
  然后我就继续请教,B同学安装成功了并把它搜到的的教程发给了我,终于在按照他发的教程安第二遍的时候,我终于成功了!!!texstudio终于识别了!
  
  之后我按照老师发的那两个链接里的内容学习git和latex的功能和使用,我发现在texstudio里面要插入图片的时候最好将图片在Tex course文件夹里命名成英文,否则会报错呢,还有就是特殊字符要注意前面加转义字符,竖线符号显示成横线,我也试着加转义符号,然而无果,于是我试着把它当成数学公式,前后加美元符号就可以啦!通过这次实验我了解了Git和latex的功能和使用,收获颇多,继续努力,天天向上!
  \section{参考链接}
  \href{https://b23.tv/G6xih3X}{https://b23.tv/G6xih3X}
  
  \href{https://b23.tv/UmgkBTp}{https://b23.tv/UmgKBTp}
  
  \href{https://blog.csdn.net/weixin_43888891/article/details/131546020}{https://blog.csdn.net/weixin\_43888891/article/details/131546020}
  
  \href{https://git-scm.com/download/win}{https://git-scm.com/download/win}
  
  \href{https://oi-wiki.org/tools/latex/}{https://oi-wiki.org/tools/latex/}
  
  \href{https://missing-semester-cn.github.io/}{https://missing-semester-cn.github.io/}
\end{document}