\documentclass[a4paper ,12pt]{article}
\usepackage{ctex}
\usepackage{hyperref}
\usepackage{graphicx}

\begin{document}
	
	\title{系统开发工具基础课程实验报告}
	\author{姓名:张誉馨}
	\date{\today}
	\maketitle
	
	\pagenumbering{roman}
	\tableofcontents
	\newpage
	\pagenumbering{arabic}
	
	\section{练习内容和结果}
	
	\subsection{Shell工具和脚本}
	1.阅读 $man${ }$ls$ ,然后使用 $ls$ 命令进行如下操作:
	
	2.所有文件(包括隐藏文件):$-a$
	
	3.文件打印以人类可以理解的格式输出 (例如,使用 454M 而不是 454279954):$-h$
	
	4.文件以最近访问顺序排序:$-t$
	
	5.以彩色文本显示输出结果:$--color=auto$
	
	\begin{figure}[h]
		\centering
		\includegraphics[width=0.5\textwidth]{p23}
		\caption{man ls}
	\end{figure}
	
	\begin{figure}[h]
		\centering
		\includegraphics[width=0.5\textwidth]{p22}
		\caption{使用ls命令进行操作}
	\end{figure}
	
    6.编写两个 bash 函数 marco 和 polo 执行下面的操作。 每当你执行 marco 时,当前的工作目录应当以某种形式保存,当执行 polo 时,无论现在处在什么目录下,都应当 cd 回到当时执行 marco 的目录。 为了方便 debug,你可以把代码写在单独的文件 marco.sh 中,并通过 source marco.sh 命令,(重新)加载函数。
		\begin{figure}[h]
		\centering
		\includegraphics[width=0.5\textwidth]{p24}
		\caption{两个bash函数marco和polo}
	\end{figure}
	
		\begin{figure}[h]
		\centering
		\includegraphics[width=0.5\textwidth]{p25}
		\caption{source marco.sh 命令}
	\end{figure}
	
	7.假设您有一个命令,它很少出错。因此为了在出错时能够对其进行调试,需要花费大量的时间重现错误并捕获输出。 编写一段bash脚本,运行如下的脚本直到它出错,将它的标准输出和标准错误流记录到文件,并在最后输出所有内容。 加分项:报告脚本在失败前共运行了多少次。
    	\begin{figure}[h]
    	\centering
    	\includegraphics[width=0.5\textwidth]{p26}
    	\caption{编写的bash脚本}
    \end{figure}
	
		\begin{figure}[h]
		\centering
		\includegraphics[width=0.5\textwidth]{p27}
		\caption{报告脚本在失败前运行多少次}
	\end{figure}
	
	8.本节课我们讲解的 find 命令中的 -exec 参数非常强大,它可以对我们查找的文件进行操作。 如果我们要对所有文件进行操作呢?例如创建一个zip压缩文件?我们已经知道,命令行可以从参数或标准输入接受输入。在用管道连接命令时,我们将标准输出和标准输入连接起来,但是有些命令,例如tar 则需要从参数接受输入。这里我们可以使用xargs 命令,它可以使用标准输入中的内容作为参数。 例如 ls | xargs rm 会删除当前目录中的所有文件。您的任务是编写一个命令,它可以递归地查找文件夹中所有的HTML文件,并将它们压缩成zip文件。注意,即使文件名中包含空格,您的命令也应该能够正确执行(提示:查看 xargs的参数-d)译注:MacOS 上的 xargs没有-d,查看这个issue
	
	如果您使用的是 MacOS,请注意默认的 BSD find 与GNU coreutils 中的是不一样的。你可以为find添加-print0选项,并为xargs添加-0选项。作为 Mac 用户,您需要注意 mac 系统自带的命令行工具和 GNU 中对应的工具是有区别的;如果你想使用 GNU 版本的工具,也可以使用 brew 来安装。
	(1)首先创建所需的文件
	
	(2)执行find命令
	
		\begin{figure}[h]
		\centering
		\includegraphics[width=0.5\textwidth]{p28}
		\caption{创建所需的文件后执行find命令}
	\end{figure}
	9.(进阶)编写一个命令或脚本递归的查找文件夹中最近使用的文件。更通用的做法,你可以按照最近的使用时间列出文件吗?
	
	find . -type f -print0 $|$ xargs -0 ls -lt $|$ head -1
    	\begin{figure}[h]
    	\centering
    	\includegraphics[width=0.5\textwidth]{p29}
    	\caption{编写脚本按照最近的使用时间列出文件}
    \end{figure}
	
	 \subsection{编辑器Vim}
	 1.完成 vimtutor。备注:它在一个 80x24(80 列,24 行) 终端窗口看起来效果最好。
	 vimtutor
	 
	 2.下载我们的vimrc,然后把它保存到 \~{}/.vimrc。 通读这个注释详细的文件 (用 Vim!), 然后观察 Vim 在这个新的设置下看起来和使用起来有哪些细微的区别。
	 
	 先下载下来这个文件,之后拖入虚拟机里就复制成功了,然后vim它
	 
	 
	 3.安装和配置一个插件: ctrlp.vim.
	 用 mkdir -p ~/.vim/pack/vendor/start 创建插件文件夹
	 下载这个插件: cd ~/.vim/pack/vendor/start; git clone https://github.com/ctrlpvim/ctrlp.vim
	 下载后需要在~/.vimrc 中添加如下设置,参考这里
	 set runtimepath\^=~/.vim/pack/vendor/start/ctrlp.vim
	 
	 4.请阅读这个插件的文档。 尝试用 CtrlP 来在一个工程文件夹里定位一个文件, 打开 Vim, 然后用 Vim 命令控制行开始 :CtrlP.
	 
	 5.自定义 CtrlP: 添加 configuration 到你的 ~/.vimrc 来用按 Ctrl-P 打开 CtrlP
	 
	 
	 \subsection{数据整理}
	 	
	 \section{解题感悟} 
	 
	 \section{github链接}
	 
	 
\end{document}